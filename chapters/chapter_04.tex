\begin{savequote}[90mm]
  {\QuoteFont Praesent metus ligula, auctor vitae, lacinia sed.}
\qauthor{- Chao Chao}
\end{savequote}

\chapter{Example of Math}
\label{chap:chapter_04}

Mauris mollis condimentum risus. Integer ipsum. Quisque
malesuada, erat ac dictum pulvinar, magna nisl fermentum ligula,
quis euismod mauris felis non diam. Nullam sapien turpis, rutrum
vel, condimentum ac, bibendum vulputate, nulla. Vestibulum tortor
ipsum, fermentum egestas, placerat ut, vulputate et, wisi. Aliquam
erat volutpat. Each element is defined as $t_{seek}$,$t_{rot}$, and
$t_{trans}$, respectively. Donec consequat, ligula sit amet tincidunt aliquam,
nunc lorem sagittis nunc, a ullamcorper erat ante ac felis. Donec
eleifend. Nullam quam leo, lobortis non, condimentum at, tempus
consectetuer, orci. Quisque ut lorem.  Vestibulum ante ipsum primis
in faucibus orci luctus et ultrices posuere cubilia Curae; Donec
porta, libero eget feugiat posuere, felis arcu pulvinar odio, vel
dapibus enim dui nec turpis. \begin{equation}
\label{eq:attf}
t_{HDD}(f_{n}) = t_{seek}(f_{n}) +t_{rot}(f_{n}) + t_{trans}(f_{n})
\end{equation}
Suspendisse porta, dolor sed fringilla
ultrices, augue mauris gravida dolor, vel sollicitudin magna dui sit
amet nunc. 

 
Donec nisl. Lorem ipsum dolor sit amet, consectetuer adipiscing elit. More
detailed description of the expressions is found in \cite{miner,CloSpan}. Let
$F=\{ f_{1}, ~ f_{2}, ..., ~ f_{n} \}$ be a set of all files accessed during
launch of an application.  A sequence 
$\sigma = < s_{1}, ~ s_{2}, ..., ~ s_{i} >$ 
is an ordered list and is a subset of $F$ such that 
$\{ s_{n} \sqsubseteq F \}$. 
Given two sequences 
$\alpha = < a_{1}, ~ a_{2}, ..., ~ a_{n} >$ and 
$\beta = < b_{1}, ~ b_{2}, ..., ~ b_{m} >$, sub-sequence is defined as follows:
Sequence  $\alpha$ is a sub-sequence of another sequence $\beta$, if and only
if there exist $i_{1}, ~ i_{2}, \ldots, ~ i_{m}$ such that 
$1 \le i_{1} < i_{2} < \ldots < i_{m} \le n$  and 
$a_{1} \subseteq b_{i1},  ~ a_{2} \subseteq b_{i2} , \ldots and  a_{m} \subseteq b_{im}$. 
If $\alpha \neq \beta$ then it is denoted as $ \alpha \subset \beta $. 
If files in sequence $\alpha$ are in the
same order as sequence $\beta$, then we call $\beta$ a super-sequence of
$\alpha$.  A sequence database, 
$D = \{ \sigma_{1}, ~ \sigma_{2}, ..., ~ \sigma_{i} \}$, 
is a set of sequences, and $|D|$ is the number of sequences in
the database $D$. The support of a sequence $\alpha$ in $D$ is the number of
sequences of $D$ which contains $\alpha$ as a subsequence. We define frequent
sub-sequence as sequences with the support greater than or equal to $min\_sup$,
where $min\_sup$ is the minimum support threshold.

{\bfseries \noindent Example 1.} Suppose a database $D$ with $|D| = 5$ has the
following set of sequences $\{ ab, ~ acde, ~ bcdf, ~ abcd, ~ abcf \}$. Then, 2
is the support of $abc$ and 3 is the support of $ac$ in $D$. Suppose 
$min\_sup = 3$, then $abc$ is not a frequent sub-sequence but $ac$ is a frequent
sub-sequence. $\blacksquare$


 

\lipsum[1-4]
